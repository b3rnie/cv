%%%%%%%%%%%%%%%%%%%%%%%%%%%% Document Setup %%%%%%%%%%%%%%%%%%%%%%%%%%%%

% Don't like 10pt? Try 11pt or 12pt
\documentclass[10pt]{article}

% \usepackage{times}
\usepackage{mathptmx}
\usepackage[scaled=.90]{helvet}
\usepackage{courier}

\renewcommand{\familydefault}{\sfdefault}

%% This makes font encoding use 8 bits instead of the default 7.
\usepackage[T1]{fontenc}

% The OCR software also has a hard time with italics. These commands get
% rid of the two common ways to italicize text in LaTeX. Get rid of them
% to turn italics back on.
\renewcommand\emph[1]{#1}
\renewcommand\textit[1]{\underline{#1}}

% This is a helpful package that puts math inside length specifications
\usepackage{calc}

% Layout: Puts the section titles on left side of page
\reversemarginpar

\usepackage[paper=a4paper,
            %includefoot, % Uncomment to put page number above margin
            marginparwidth=30.5mm,    % Length of section titles
            marginparsep=1.5mm,       % Space between titles and text
            margin=25mm,              % 25mm margins
            includemp]{geometry}

%% More layout: Get rid of indenting throughout entire document
\setlength{\parindent}{0in}

\usepackage[shortlabels]{enumitem}

\makeatletter
\newlength{\bibhang}
\setlength{\bibhang}{1em}
\newlength{\bibsep}
 {\@listi \global\bibsep\itemsep \global\advance\bibsep by\parsep}
\newlist{bibsection}{itemize}{3}
\setlist[bibsection]{label=,leftmargin=\bibhang,%
        itemindent=-\bibhang,
        itemsep=\bibsep,parsep=\z@,partopsep=0pt,
        topsep=0pt}
\newlist{bibenum}{enumerate}{3}
\setlist[bibenum]{label=[\arabic*],resume,leftmargin={\bibhang+\widthof{[999]}},%
        itemindent=-\bibhang,
        itemsep=\bibsep,parsep=\z@,partopsep=0pt,
        topsep=0pt}
\let\oldendbibenum\endbibenum
\def\endbibenum{\oldendbibenum\vspace{-.6\baselineskip}}
\let\oldendbibsection\endbibsection
\def\endbibsection{\oldendbibsection\vspace{-.6\baselineskip}}
\makeatother

\usepackage{fancyhdr,lastpage}
\pagestyle{fancy}
%\pagestyle{empty}      % Uncomment this to get rid of page numbers
\fancyhf{}\renewcommand{\headrulewidth}{0pt}
\fancyfootoffset{\marginparsep+\marginparwidth}
\newlength{\footpageshift}
\setlength{\footpageshift}
          {0.5\textwidth+0.5\marginparsep+0.5\marginparwidth-2in}

%%%% PAGES 2--9 NUMBERING:
%% These two lines put page number in upper-right corner of pages 2--end
\rhead{Jensen-Urstad, p.~\arabic{page} of \protect\pageref*{LastPage}}   % +LP
%\rhead{Pavlic, p.~\arabic{page}}                                 % -LP

%% These lines put page number in bottom (center) of pages 2--end
%\lfoot{\hspace{\footpageshift}%
%       \parbox{4in}{\, \hfill %
%                    \arabic{page} of \protect\pageref*{LastPage} % +LP
%%                    \arabic{page}                               % -LP
%                    \hfill \,}}
%%%% END PAGE 2--9 NUMBERING

%%%% PAGE 1 NUMBERING:
\makeatletter
\let\oldps@plain\ps@plain
\renewcommand{\ps@plain}{\oldps@plain%
\renewcommand{\@evenfoot}{\hspace*{-\footpageshift}\hfil %
    p.~\arabic{page} of \protect\pageref*{LastPage} % +LP
%    p.~\arabic{page}                               % -LP
    \hfil}%
\renewcommand{\@oddfoot}{\@evenfoot}}
\makeatother
%%%% END PAGE 1 NUMBERING

% Finally, give us PDF bookmarks and colored links
\usepackage{color,hyperref}
\definecolor{darkblue}{rgb}{0.0,0.0,0.3}
\hypersetup{breaklinks,colorlinks,
            linkcolor=black,urlcolor=black,
            anchorcolor=black,citecolor=black,
            %linkcolor=darkblue,urlcolor=darkblue,
            %anchorcolor=darkblue,citecolor=darkblue,
            %draft
            }

\newcommand{\makeheading}[2][]%
        {\hspace*{-\marginparsep minus \marginparwidth}%
         \begin{minipage}[t]{\textwidth+\marginparwidth+\marginparsep}%
             {\large \bfseries #2 \hfill #1}\\[-0.15\baselineskip]%
                 \rule{\columnwidth}{1pt}%
         \end{minipage}}

\renewcommand{\section}[1]{\pagebreak[3]%
    \vspace{1.3\baselineskip}%
    \phantomsection\addcontentsline{toc}{section}{#1}%
    \noindent\llap{\scshape\smash{\parbox[t]{\marginparwidth}{\hyphenpenalty=10000\raggedright #1}}}%
    \vspace{-\baselineskip}\par}

\newcommand*\fixendlist[1]{%
    \expandafter\let\csname preFixEndListend#1\expandafter\endcsname\csname end#1\endcsname
    \expandafter\def\csname end#1\endcsname{\csname preFixEndListend#1\endcsname\vspace{-0.6\baselineskip}}}

% These macros help ensure that items in outer-type lists do not get
% separated from the next line by a page break
% (they are used by lists below)
\let\originalItem\item
\newcommand*\fixouterlist[1]{%
    \expandafter\let\csname preFixOuterList#1\expandafter\endcsname\csname #1\endcsname
    \expandafter\def\csname #1\endcsname{\csname preFixOuterList#1\endcsname\let\oldItem\item\def\item{\pagebreak[2]\oldItem}}
    \expandafter\let\csname preFixOuterListend#1\expandafter\endcsname\csname end#1\endcsname
    \expandafter\def\csname end#1\endcsname{\let\item\oldItem\csname preFixOuterListend#1\endcsname}}
\newcommand*\fixinnerlist[1]{%
    \expandafter\let\csname preFixInnerList#1\expandafter\endcsname\csname #1\endcsname
    \expandafter\def\csname #1\endcsname{\let\oldItem\item\let\item\originalItem\csname preFixInnerList#1\endcsname}
    \expandafter\let\csname preFixInnerListend#1\expandafter\endcsname\csname end#1\endcsname
    \expandafter\def\csname end#1\endcsname{\csname preFixInnerListend#1\endcsname\let\item\oldItem}}

% An itemize-style list with lots of space between items
%
% Usage:
%   \begin{outerlist}
%       \item ...    % (or \item[] for no bullet)
%   \end{outerlist}
\newlist{outerlist}{itemize}{3}
    \setlist[outerlist]{label=\enskip\textbullet,leftmargin=*}
    \fixendlist{outerlist}
    \fixouterlist{outerlist}

% An environment IDENTICAL to outerlist that has better pre-list spacing
% when used as the first thing in a \section
%
% Usage:
%   \begin{lonelist}
%       \item ...    % (or \item[] for no bullet)
%   \end{lonelist}
\newlist{lonelist}{itemize}{3}
    \setlist[lonelist]{label=\enskip\textbullet,leftmargin=*,partopsep=0pt,topsep=0pt}
    \fixendlist{lonelist}
    \fixouterlist{lonelist}

% An itemize-style list with little space between items
%
% Usage:
%   \begin{innerlist}
%       \item ...    % (or \item[] for no bullet)
%   \end{innerlist}
\newlist{innerlist}{itemize}{3}
    \setlist[innerlist]{label=\enskip\textbullet,leftmargin=*,parsep=0pt,itemsep=0pt,topsep=0pt,partopsep=0pt}
    \fixinnerlist{innerlist}

% An environment IDENTICAL to innerlist that has better pre-list spacing
% when used as the first thing in a \section
%
% Usage:
%   \begin{loneinnerlist}
%       \item ...    % (or \item[] for no bullet)
%   \end{loneinnerlist}
\newlist{loneinnerlist}{itemize}{3}
    \setlist[loneinnerlist]{label=\enskip\textbullet,leftmargin=*,parsep=0pt,itemsep=0pt,topsep=0pt,partopsep=0pt}
    \fixendlist{loneinnerlist}
    \fixinnerlist{loneinnerlist}

%%% EXTRA SPACE

% To add some paragraph space between lines.
% This also tells LaTeX to preferably break a page on one of these gaps
% if there is a needed pagebreak nearby.
\newcommand{\blankline}{\quad\pagebreak[3]}
\newcommand{\halfblankline}{\quad\vspace{-0.5\baselineskip}\pagebreak[3]}

%%% FORMATTING MACROS

% Uses hyperref to link DOI
\newcommand\doilink[1]{\href{http://dx.doi.org/#1}{#1}}
\newcommand\doi[1]{doi:\doilink{#1}}

% For \url{SOME_URL}, links SOME_URL to the url SOME_URL
\providecommand*\url[1]{\href{#1}{#1}}
% Same as above, but pretty-prints SOME_URL in teletype fixed-width font
\renewcommand*\url[1]{\href{#1}{\texttt{#1}}}

% For \email{ADDRESS}, links ADDRESS to the url mailto:ADDRESS
\providecommand*\email[1]{\href{mailto:#1}{#1}}
% Same as above, but pretty-prints ADDRESS in teletype fixed-width font
%\renewcommand*\email[1]{\href{mailto:#1}{\texttt{#1}}}

%\providecommand\BibTeX{{\rm B\kern-.05em{\sc i\kern-.025em b}\kern-.08em
%    T\kern-.1667em\lower.7ex\hbox{E}\kern-.125emX}}
%\providecommand\BibTeX{{\rm B\kern-.05em{\sc i\kern-.025em b}\kern-.08em
%    \TeX}}
\providecommand\BibTeX{{B\kern-.05em{\sc i\kern-.025em b}\kern-.08em
    \TeX}}
\providecommand\Matlab{\textsc{Matlab}}

% Custom hyphenation rules for words that LaTeX has trouble with
\hyphenation{bio-mim-ic-ry bio-in-spi-ra-tion re-us-a-ble pro-vid-er}

%%%%%%%%%%%%%%%%%%%%%%%% End Helper Commands %%%%%%%%%%%%%%%%%%%%%%%%%%%

%%%%%%%%%%%%%%%%%%%%%%%%% Begin CV Document %%%%%%%%%%%%%%%%%%%%%%%%%%%%

\begin{document}
\thispagestyle{plain}
\makeheading[\emph{Curriculum vitae}]{Björn Jensen-Urstad}

\section{Contact Information}

% NOTE: Mind where the & separators and \\ breaks are in the following
%       table. Table is one row made up of three parboxes. The left
%       parbox has address info, the middle parbox has a vertical bar,
%       and the right parbox has phone and electronic contact
%       information.
%
% MACROS: \rcollength is the width of the right column of the table
%             (adjust it to your liking; default is 1.85in).
%         \spacewidth is width of area between left and right boxes.
%         \spacechar is character used to produce perforated vertical
%             boundary between boxes.
%
\newlength{\rcollength}\setlength{\rcollength}{2.5in}%
\newlength{\spacewidth}\setlength{\spacewidth}{20pt}
\newcommand\spacechar{$|$}
%
\begin{tabular}[t]{@{}p{\textwidth-\rcollength-\spacewidth}@{}p{\spacewidth}@{}p{\rcollength}}%

% Address box
\parbox{\textwidth-\rcollength-\spacewidth}{%
Björn Jensen-Urstad\\
Skeppargatan 96\\
115 30 Stockholm\\
Sweden}

% Cheesy perforated vertical bar between boxes
% Shorten by removing \spacechar's
& \parbox{\spacewidth}{\centering \spacechar\\\spacechar\\\spacechar\\\spacechar\\\spacechar} &

% Non-snail-mail contact information
\parbox{\rcollength}{%
% \emph{Mobile:} +46-70-0222948\\
\emph{E-mail:} \email{jobs@bolg.se}\\
%%\emph{WWW:} \href{http://www.bernie.com/}{http://www.bolg.se/}\\
\emph{GitHub:} \href{https://github.com/b3rnie/}{https://github.com/b3rnie/}\\
\emph{Linkedin:} \href{https://www.linkedin.com/in/bjornjensenurstad}{linkedin.com/in/bjornjensenurstad}}
\end{tabular}

\section{About}

Software engineer specializing in backend, big data, cloud and robust scalable systems.

I have worked at small startups and large companies. Building software from scratch, deve\-loped or migrated mature complex systems.
I enjoy finding easy solutions to complex problems.

\halfblankline

My core skills are:
\begin{innerlist}
  \item Scala, Python
  \item Architecture and software design
  \item Devops
  \item Databases
\end{innerlist}
\section{Availability}

\begin{loneinnerlist}
  %%\item Start time is negotiable; may be possible to start immediately
\item Fully remote or hybrid. Geographic location is flexible but there is a preference for Stock\-holm, Sweden
\end{loneinnerlist}

\section{Professional Experience}

\href{https://www.bonniernews.se/}{\textbf{Bonnier News}},
Stockholm

\begin{outerlist}
\item[] \textit{Engineer}
  \hfill \textbf{December 2019-}
  \begin{innerlist}
  \item
        Rewrote news-publishing pipeline for Dagens Industri and removed a large legacy CRM called EPi-server. Migrated on-prem databases and services to k8s and Google Cloud.
  \item
        Built data-pipelines on Apache/Airflow, Dataflow and BigQuery.
  \end{innerlist}
\end{outerlist}
\halfblankline

\href{http://campanja.com/}{\textbf{Campanja}},
Stockholm

\begin{outerlist}
\item[] \textit{Data Engineer}
	\hfill \textbf{April 2019 to November 2019}
	\begin{innerlist}
	\item
        Worked as a data scientist and engineer. Did research, experiments, notebooks and built products out of our findings.
	\end{innerlist}
\end{outerlist}
\halfblankline

\href{http://gpsgate.com/}{\textbf{GpsGate}},
Stockholm

\begin{outerlist}
\item[] \textit{Engineer}
  \hfill \textbf{January 2017 to April 2019}
  \begin{innerlist}
  \item
    Built a metrics pipeline to collect customer server statistics and created real-time dashboards.
  \item
    Worked on scaling a .NET system.
  \item
    Implemented a subscription based license model/billing system where servers report usage and are charged based on
    that. Resulted in a more predictable revenue stream when customers pay monthly instead of yearly.
  \end{innerlist}
\end{outerlist}

\halfblankline

\href{http://www.trueaccord.com/}{\textbf{TrueAccord}},
Stockholm, Sweden/San Francisco, US

Remote work from Stockholm with periodic visits to San Francisco
\begin{outerlist}
\item[] \textit{Engineer}
  \hfill \textbf{September 2014 to November 2016}
  \begin{innerlist}
  \item
      Worked on the core infrastructure, major contributions included developing a scheduling service to contact
      debtors an the optimum time, a rule system to systematically deal with the compliance and regulations
      in debt collection and automatic dispute handling.
  \end{innerlist}
\end{outerlist}

\halfblankline

\textbf{Self employed},
Stockholm, Sweden
\begin{outerlist}
\item[] \textit{Consulting}
  \hfill \textbf{August 2013-}
  \begin{innerlist}
    \item
      Designed, implemented and maintained a batch processing system. The system was built to scale and be able to
      handle terrabytes of data each day.
  \end{innerlist}
\end{outerlist}

\halfblankline

\href{http://www.kivra.se/}{\textbf{Kivra AB}},
Stockholm, Sweden
\begin{outerlist}
\item[] \textit{Engineer}
  \hfill \textbf{March 2013 to September 2013}
  \begin{innerlist}
    \item
      Architecture, development and maintenance of services.
  \end{innerlist}
\end{outerlist}

\halfblankline

\href{http://www.klarna.com/}{\textbf{Klarna AB}},
Stockholm, Sweden
\begin{outerlist}
\item[] \textit{Engineer}%
  \hfill \textbf{March 2010 to January 2013}
  \begin{innerlist}
  \item
    Extended an existing transactional database (Mnesia) with features to allow parts of the data to be
    shared in an eventual consistent manner. This involved writing a framework to allow easy transformation
    of keys/values between two very different databases with different availability requirements, adding vector clocks
    to an existing database schema and mechanisms to transport/sync data.
    The databases are the core of a business with a monthly revenue of 200 million USD.
  \item
    Design and implementation of the alpha version of Klarna's checkout.
  \item
    Rewrote and replaced Klarna's old fraud checking system with a DSL accessible through a web GUI.
    This gave staff more freedom to express what they want and the possibility to change rules without involving developers.
  \end{innerlist}
\end{outerlist}

\halfblankline

\href{http://www.posten.se/}{\textbf{Posten AB}},
Stockholm, Sweden
\begin{outerlist}
\item[] \textit{Mailman}%
  \hfill \textbf{2001 to 2002 and 2004 to 2005}
\end{outerlist}

\halfblankline

\section{Education}

\href{http://www.kth.com/}{\textbf{Royal Institute of Technology}},
Stockholm, Sweden
\begin{outerlist}

\item[] B.S.,
        \href{http://www.kth.se/}
             {Computer Science}, 2006-2010, specializing in software
             development

\item[] MS.S.,
        \href{http://www.kth.se/}
             {Computer Science}, 2002-2004, four semesters out of 9

\end{outerlist}

\halfblankline

\textbf{Hersby gymnasium}, Lidingö, Sweden
\begin{outerlist}
\item[] Upper secondary school, 1997-2000
\end{outerlist}

\vspace{0.1in}

\section{Opensource}

Worked on several open-source projects (see GitHub), including:
\begin{innerlist}
  \item Crontab implementation in Erlang
  \item Clone of Apache Kafka in Scala using Netty
  \item Bittorrent client and DHT implementation in Ruby
  \item Unique id generator in Erlang
  \item Various components for integrating with Apache ZooKeeper
  \item NNTP fetcher/assembler in Perl
  \item Socks 4/5 proxy in C++, single thread and nonblocking IO
\end{innerlist}

\halfblankline

\section{Skills}
%
\begin{innerlist}
    \item Software development in Scala, Python, Ruby, JavaScript and Erlang
    \item Many frameworks techniques including scrum, kanban, tdd and property based testing
    \item Deep understanding of architecture, large systems, databases and scaling
    \item Linux user since the 90s, I have been using most of the popular server software in production at some point
    \item Cloud (AWS, GCP)
    \item Protocols, written clients and servers for most of the common ones. Including my own TCP/IP stack on a 8-bit
    PIC microcontroller with a webserver and bittorrent tracker
    \item PCP design and creation, digital electronics and low-power embedded systems
\end{innerlist}

\halfblankline


\section{Other}
In my spare time I enjoy spending time with my girlfriend and our little dog. I also like training and outdoor activites.
I train Muay Thai and do rock climbing.


% \section{References}
% 
% Available upon request.

\vspace{0.1in}
%% \section{More Information}
%% 
%% More information and auxiliary documents can be found at\\%
%% \url{http://www.bolg.se/}.

\end{document}
