%%%%%%%%%%%%%%%%%%%%%%%%%%%% Document Setup %%%%%%%%%%%%%%%%%%%%%%%%%%%%

% Don't like 10pt? Try 11pt or 12pt
\documentclass[10pt]{article}

% \usepackage{times}
\usepackage{mathptmx}
\usepackage[scaled=.90]{helvet}
\usepackage{courier}

\renewcommand{\familydefault}{\sfdefault}

%% This makes font encoding use 8 bits instead of the default 7.
\usepackage[T1]{fontenc}

% The OCR software also has a hard time with italics. These commands get
% rid of the two common ways to italicize text in LaTeX. Get rid of them
% to turn italics back on.
\renewcommand\emph[1]{#1}
\renewcommand\textit[1]{\underline{#1}}

% This is a helpful package that puts math inside length specifications
\usepackage{calc}

% Layout: Puts the section titles on left side of page
\reversemarginpar

\usepackage[paper=a4paper,
            %includefoot, % Uncomment to put page number above margin
            marginparwidth=30.5mm,    % Length of section titles
            marginparsep=1.5mm,       % Space between titles and text
            margin=25mm,              % 25mm margins
            includemp]{geometry}

%% More layout: Get rid of indenting throughout entire document
\setlength{\parindent}{0in}

\usepackage[shortlabels]{enumitem}

\makeatletter
\newlength{\bibhang}
\setlength{\bibhang}{1em}
\newlength{\bibsep}
 {\@listi \global\bibsep\itemsep \global\advance\bibsep by\parsep}
\newlist{bibsection}{itemize}{3}
\setlist[bibsection]{label=,leftmargin=\bibhang,%
        itemindent=-\bibhang,
        itemsep=\bibsep,parsep=\z@,partopsep=0pt,
        topsep=0pt}
\newlist{bibenum}{enumerate}{3}
\setlist[bibenum]{label=[\arabic*],resume,leftmargin={\bibhang+\widthof{[999]}},%
        itemindent=-\bibhang,
        itemsep=\bibsep,parsep=\z@,partopsep=0pt,
        topsep=0pt}
\let\oldendbibenum\endbibenum
\def\endbibenum{\oldendbibenum\vspace{-.6\baselineskip}}
\let\oldendbibsection\endbibsection
\def\endbibsection{\oldendbibsection\vspace{-.6\baselineskip}}
\makeatother

\usepackage{fancyhdr,lastpage}
\pagestyle{fancy}
%\pagestyle{empty}      % Uncomment this to get rid of page numbers
\fancyhf{}\renewcommand{\headrulewidth}{0pt}
\fancyfootoffset{\marginparsep+\marginparwidth}
\newlength{\footpageshift}
\setlength{\footpageshift}
          {0.5\textwidth+0.5\marginparsep+0.5\marginparwidth-2in}

%%%% PAGES 2--9 NUMBERING:
%% These two lines put page number in upper-right corner of pages 2--end
\rhead{Jensen-Urstad, p.~\arabic{page} of \protect\pageref*{LastPage}}   % +LP
%\rhead{Pavlic, p.~\arabic{page}}                                 % -LP

%% These lines put page number in bottom (center) of pages 2--end
%\lfoot{\hspace{\footpageshift}%
%       \parbox{4in}{\, \hfill %
%                    \arabic{page} of \protect\pageref*{LastPage} % +LP
%%                    \arabic{page}                               % -LP
%                    \hfill \,}}
%%%% END PAGE 2--9 NUMBERING

%%%% PAGE 1 NUMBERING:
\makeatletter
\let\oldps@plain\ps@plain
\renewcommand{\ps@plain}{\oldps@plain%
\renewcommand{\@evenfoot}{\hspace*{-\footpageshift}\hfil %
    p.~\arabic{page} of \protect\pageref*{LastPage} % +LP
%    p.~\arabic{page}                               % -LP
    \hfil}%
\renewcommand{\@oddfoot}{\@evenfoot}}
\makeatother
%%%% END PAGE 1 NUMBERING

% Finally, give us PDF bookmarks and colored links
\usepackage{color,hyperref}
\definecolor{darkblue}{rgb}{0.0,0.0,0.3}
\hypersetup{breaklinks,colorlinks,
            linkcolor=black,urlcolor=black,
            anchorcolor=black,citecolor=black,
            %linkcolor=darkblue,urlcolor=darkblue,
            %anchorcolor=darkblue,citecolor=darkblue,
            %draft
            }

\newcommand{\makeheading}[2][]%
        {\hspace*{-\marginparsep minus \marginparwidth}%
         \begin{minipage}[t]{\textwidth+\marginparwidth+\marginparsep}%
             {\large \bfseries #2 \hfill #1}\\[-0.15\baselineskip]%
                 \rule{\columnwidth}{1pt}%
         \end{minipage}}

\renewcommand{\section}[1]{\pagebreak[3]%
    \vspace{1.3\baselineskip}%
    \phantomsection\addcontentsline{toc}{section}{#1}%
    \noindent\llap{\scshape\smash{\parbox[t]{\marginparwidth}{\hyphenpenalty=10000\raggedright #1}}}%
    \vspace{-\baselineskip}\par}

\newcommand*\fixendlist[1]{%
    \expandafter\let\csname preFixEndListend#1\expandafter\endcsname\csname end#1\endcsname
    \expandafter\def\csname end#1\endcsname{\csname preFixEndListend#1\endcsname\vspace{-0.6\baselineskip}}}

% These macros help ensure that items in outer-type lists do not get
% separated from the next line by a page break
% (they are used by lists below)
\let\originalItem\item
\newcommand*\fixouterlist[1]{%
    \expandafter\let\csname preFixOuterList#1\expandafter\endcsname\csname #1\endcsname
    \expandafter\def\csname #1\endcsname{\csname preFixOuterList#1\endcsname\let\oldItem\item\def\item{\pagebreak[2]\oldItem}}
    \expandafter\let\csname preFixOuterListend#1\expandafter\endcsname\csname end#1\endcsname
    \expandafter\def\csname end#1\endcsname{\let\item\oldItem\csname preFixOuterListend#1\endcsname}}
\newcommand*\fixinnerlist[1]{%
    \expandafter\let\csname preFixInnerList#1\expandafter\endcsname\csname #1\endcsname
    \expandafter\def\csname #1\endcsname{\let\oldItem\item\let\item\originalItem\csname preFixInnerList#1\endcsname}
    \expandafter\let\csname preFixInnerListend#1\expandafter\endcsname\csname end#1\endcsname
    \expandafter\def\csname end#1\endcsname{\csname preFixInnerListend#1\endcsname\let\item\oldItem}}

% An itemize-style list with lots of space between items
%
% Usage:
%   \begin{outerlist}
%       \item ...    % (or \item[] for no bullet)
%   \end{outerlist}
\newlist{outerlist}{itemize}{3}
    \setlist[outerlist]{label=\enskip\textbullet,leftmargin=*}
    \fixendlist{outerlist}
    \fixouterlist{outerlist}

% An environment IDENTICAL to outerlist that has better pre-list spacing
% when used as the first thing in a \section
%
% Usage:
%   \begin{lonelist}
%       \item ...    % (or \item[] for no bullet)
%   \end{lonelist}
\newlist{lonelist}{itemize}{3}
    \setlist[lonelist]{label=\enskip\textbullet,leftmargin=*,partopsep=0pt,topsep=0pt}
    \fixendlist{lonelist}
    \fixouterlist{lonelist}

% An itemize-style list with little space between items
%
% Usage:
%   \begin{innerlist}
%       \item ...    % (or \item[] for no bullet)
%   \end{innerlist}
\newlist{innerlist}{itemize}{3}
    \setlist[innerlist]{label=\enskip\textbullet,leftmargin=*,parsep=0pt,itemsep=0pt,topsep=0pt,partopsep=0pt}
    \fixinnerlist{innerlist}

% An environment IDENTICAL to innerlist that has better pre-list spacing
% when used as the first thing in a \section
%
% Usage:
%   \begin{loneinnerlist}
%       \item ...    % (or \item[] for no bullet)
%   \end{loneinnerlist}
\newlist{loneinnerlist}{itemize}{3}
    \setlist[loneinnerlist]{label=\enskip\textbullet,leftmargin=*,parsep=0pt,itemsep=0pt,topsep=0pt,partopsep=0pt}
    \fixendlist{loneinnerlist}
    \fixinnerlist{loneinnerlist}

%%% EXTRA SPACE

% To add some paragraph space between lines.
% This also tells LaTeX to preferably break a page on one of these gaps
% if there is a needed pagebreak nearby.
\newcommand{\blankline}{\quad\pagebreak[3]}
\newcommand{\halfblankline}{\quad\vspace{-0.5\baselineskip}\pagebreak[3]}

%%% FORMATTING MACROS

% Uses hyperref to link DOI
\newcommand\doilink[1]{\href{http://dx.doi.org/#1}{#1}}
\newcommand\doi[1]{doi:\doilink{#1}}

% For \url{SOME_URL}, links SOME_URL to the url SOME_URL
\providecommand*\url[1]{\href{#1}{#1}}
% Same as above, but pretty-prints SOME_URL in teletype fixed-width font
\renewcommand*\url[1]{\href{#1}{\texttt{#1}}}

% For \email{ADDRESS}, links ADDRESS to the url mailto:ADDRESS
\providecommand*\email[1]{\href{mailto:#1}{#1}}
% Same as above, but pretty-prints ADDRESS in teletype fixed-width font
%\renewcommand*\email[1]{\href{mailto:#1}{\texttt{#1}}}

%\providecommand\BibTeX{{\rm B\kern-.05em{\sc i\kern-.025em b}\kern-.08em
%    T\kern-.1667em\lower.7ex\hbox{E}\kern-.125emX}}
%\providecommand\BibTeX{{\rm B\kern-.05em{\sc i\kern-.025em b}\kern-.08em
%    \TeX}}
\providecommand\BibTeX{{B\kern-.05em{\sc i\kern-.025em b}\kern-.08em
    \TeX}}
\providecommand\Matlab{\textsc{Matlab}}

% Custom hyphenation rules for words that LaTeX has trouble with
\hyphenation{bio-mim-ic-ry bio-in-spi-ra-tion re-us-a-ble pro-vid-er}

%%%%%%%%%%%%%%%%%%%%%%%% End Helper Commands %%%%%%%%%%%%%%%%%%%%%%%%%%%

%%%%%%%%%%%%%%%%%%%%%%%%% Begin CV Document %%%%%%%%%%%%%%%%%%%%%%%%%%%%

\begin{document}
\thispagestyle{plain}
\makeheading[\emph{Curriculum vitae}]{Björn Jensen-Urstad}

\section{Contact Information}

% NOTE: Mind where the & separators and \\ breaks are in the following
%       table. Table is one row made up of three parboxes. The left
%       parbox has address info, the middle parbox has a vertical bar,
%       and the right parbox has phone and electronic contact
%       information.
%
% MACROS: \rcollength is the width of the right column of the table
%             (adjust it to your liking; default is 1.85in).
%         \spacewidth is width of area between left and right boxes.
%         \spacechar is character used to produce perforated vertical
%             boundary between boxes.
%
\newlength{\rcollength}\setlength{\rcollength}{2.5in}%
\newlength{\spacewidth}\setlength{\spacewidth}{20pt}
\newcommand\spacechar{$|$}
%
\begin{tabular}[t]{@{}p{\textwidth-\rcollength-\spacewidth}@{}p{\spacewidth}@{}p{\rcollength}}%

% Address box
\parbox{\textwidth-\rcollength-\spacewidth}{%
Björn Jensen-Urstad\\
Skeppargatan 96\\
115 30 Stockholm\\
Sweden}

% Cheesy perforated vertical bar between boxes
% Shorten by removing \spacechar's
& \parbox{\spacewidth}{\centering \spacechar\\\spacechar\\\spacechar\\\spacechar\\\spacechar} &

% Non-snail-mail contact information
\parbox{\rcollength}{%
% \emph{Mobile:} +46-70-0222948\\
\emph{E-mail:} \email{jobs@bolg.se}\\
%%\emph{WWW:} \href{http://www.bernie.com/}{http://www.bolg.se/}\\
\emph{Github:} \href{https://github.com/b3rnie/}{https://github.com/b3rnie/}}

\end{tabular}

\section{Qualifications and Interests}

Distributed systems, Scala, Python, NodeJS, Erlang/OTP, embedded systems, low power
embedded, databases, protocols, functional programming, network protocols,
eventual consistency, software archi\-tecture, message queues, etc.

\section{Availability}

\begin{loneinnerlist}
  %%\item Start time is negotiable; may be possible to start immediately
\item Geographic location is flexible, but there is preference for Stockholm, Sweden
\end{loneinnerlist}

\section{Education}

\href{http://www.kth.com/}{\textbf{Royal Institute of Technology}},
Stockholm, Sweden
\begin{outerlist}

\item[] B.S.,
        \href{http://www.kth.se/}
             {Computer Science}, 2006-2010, specializing in software
             development

\item[] MS.S.,
        \href{http://www.kth.se/}
             {Computer Science}, 2002-2004, four semesters out of 9

\end{outerlist}

\halfblankline

\textbf{Hersby gymnasium}, Lidingö, Sweden
\begin{outerlist}
\item[] Upper secondary school, 1997-2000
\end{outerlist}

\section{Professional Experience}

\href{https://www.bonniernews.se/}{\textbf{Bonnier News}},
Stockholm

\begin{outerlist}
\item[] \textit{Engineer}
  \hfill \textbf{December 2019-}
  \begin{innerlist}
  \item
        Rewrote parts of the news-publishing pipeline for di.se. Removed
        a large legacy CRM called EPi-server from all involvment in the publising
        process.
        Scala, Python, Kubernetes and ElasticSearch.
  \item
        Migrated systems and databases for di.se from physical hardware to a
        cloud-provider.
        ElasticSearch, Kubernetes, Postgres, Akamai, etc.
  \item
        Lots of NodeJS development on smaller services.
  \end{innerlist}
\end{outerlist}
\halfblankline

\href{http://campanja.com/}{\textbf{Campanja}},
Stockholm

\begin{outerlist}
\item[] \textit{Data Engineer}
	\hfill \textbf{April 2019 to November 2019}
	\begin{innerlist}
	\item
        Worked as a mixture of a data-scientist and an engineer. Did research,
        experiments, notebooks and built products out of our findings.
        Most development was done using Python, Docker and Kubernetes.
	\end{innerlist}
\end{outerlist}
\halfblankline

\href{http://gpsgate.com/}{\textbf{GpsGate}},
Stockholm

\begin{outerlist}
\item[] \textit{Engineer}
  \hfill \textbf{January 2017-}
  \begin{innerlist}
  \item
    Built a metrics pipeline to collect customer server statistics.
    Data gets aggregated continously and inserted into a time series database
    for real time dashboards. Reported data is also stored in S3 for batch
    querying.
    Implemented on AWS using S3, SQS, Athena, Prometheus, Grafana and NodeJS.
  \item
    Worked on scaling a .NET system and making it possible to run parts of it on
    a cluster if needed. Using Scala, Netty, Apache Kafka, Cassandra and Docker.
  \item
    Implemented a subscription based license model/billing system where servers
    report usage and are charged based on that. Resulted in a more predictable
    revenue stream for us since customers pay monthly instead of yearly. For
    customers it resulted in far lower startup costs.
    Implemented on AWS using MySQL (Aurora), Docker and NodeJS.
  \end{innerlist}
\end{outerlist}

\halfblankline

\href{http://www.trueaccord.com/}{\textbf{TrueAccord}},
Stockholm, Sweden/San Francisco, US

Remote work from Stockholm with periodic visits to San Francisco
\begin{outerlist}
\item[] \textit{Engineer}
  \hfill \textbf{September 2014 to November 2016}
  \begin{innerlist}
  \item
      Worked on the core infrastructure, major contributions include:
      developing a scheduling service to contact debtors an the optimum time,
      a rule system to systematically deal with the compliance and regulations
      in debt collection, automatic dispute handling, components to throttle
      and regulate (PID) rate of requests to external services.
      Most development was done using Scala.
    \item
      Numerous integrations and business features.
  \end{innerlist}
\end{outerlist}

\halfblankline

\textbf{Self employed},
Stockholm, Sweden
\begin{outerlist}
\item[] \textit{Consulting}
  \hfill \textbf{August 2013-}
  \begin{innerlist}
    \item
      Designed, implemented and maintained a batch processing system using
      Zookeeper, Riak and Erlang/OTP. The system was built to scale and be
      able to handle terrabytes of data each day.
  \end{innerlist}
\end{outerlist}

\halfblankline

\href{http://www.kivra.se/}{\textbf{Kivra AB}},
Stockholm, Sweden
\begin{outerlist}
\item[] \textit{Engineer}
  \hfill \textbf{March 2013 to September 2013}
  \begin{innerlist}
    \item
      Architecture, development and maintenance of the services that made up
      Kivra's system.
  \end{innerlist}
\end{outerlist}

\halfblankline

\href{http://www.klarna.com/}{\textbf{Klarna AB}},
Stockholm, Sweden
\begin{outerlist}
\item[] \textit{Engineer}%
  \hfill \textbf{March 2010 to January 2013}
  \begin{innerlist}
  \item
    Extended an existing transactional database (Mnesia)
    with features to allow parts of the data to be shared (read/written)
    in an eventual consistent manner.
    This involved writing a framework to allow easy transformation
    of keys/values between two very different databses with different
    availability requirements, adding vector clocks to an existing database
    schema and mechanisms to transport/sync data over RabbitMQ.
    The databases are the core of a business with a monthly revenue of
    200 million USD
  \item
    Design and implementation of the alpha version of Klarna's checkout
  \item
    Experiments with automatic fraud detection
  \item
    Rewrote and replaced Klarna's old fraud checking system with a DSL
    accessible through a web GUI.
    This gave staff more freedom to express what they want and the
    possibility to change rules without involving developers
  \item
    Lots of minor features for the business such as integrating with new
    credit check agencies.
  \end{innerlist}
\end{outerlist}

\halfblankline

\href{http://www.posten.se/}{\textbf{Posten AB}},
Stockholm, Sweden
\begin{outerlist}
\item[] \textit{Mailman}%
  \hfill \textbf{2001 to 2002 and 2004 to 2005}
\end{outerlist}

\halfblankline

\section{Memberships}
\begin{outerlist}
\item[] \textit{Housing Cooperative, Bjornligan}%
  \hfill \textbf{2007-2010}
  \begin{innerlist}
    \item Board of directors and responsible for the economy
  \end{innerlist}
\item[] \textit{EFF (Electronic Frontier Foundation)}%
\hfill \textbf{2010-}
\end{outerlist}
\vspace{0.1in}

\section{Opensource}

Written several open-source projects (see github), including:
\begin{innerlist}
  \item Crontab implementation in Erlang
  \item Clone of Apache Kafka in Scala using Netty
  \item Bittorrent client in Ruby
  \item Bittorrent DHT implementation in Ruby
  \item Unique id generator in Erlang
  \item Various components for integrating with Apache ZooKeeper
  \item NNTP fetcher/assembler in Perl
  \item Socks 4/5 proxy in C++, single thread and nonblocking IO
\end{innerlist}

\halfblankline

\section{Skills}
Software development:
%
\begin{innerlist}
    \item
      Worked with Scrum, Kanban, Git, Subversion, test driven development,
      property based testing
\end{innerlist}

\halfblankline

Computer Programming:
%
\begin{innerlist}
    \item
      Good knowledge of Scala, Java, Erlang/OTP, functional programming, Ruby
    \item
      Experience with JavaScript, C, C$+$$+$, Perl, Python
\end{innerlist}

\halfblankline

Information/Internet Technology:
%
\begin{innerlist}
    \item
      Written my own TCP/IP stack running on a 8 bit microcontroller.
      From transport layer up to application layer (Ethernet, ARP, IP, ICMP,
      TCP, UDP). Implemented a simple webserver and bittorrent tracker
      on top of it
    \item
      Written servers and or clients for Bitcoin, FTP, HTTP 1.0/1.1, SMTP,
      DNS, POP3, Jxta, NNTP, Irc, BitTorrent and Socks 4/5 as well as other
      less known protocols.
    \item
      Done web development in PHP, Ruby on Rails, NodeJS, AngularJS, Perl, CGI
\end{innerlist}

\halfblankline

Operating Systems/Software:
%
\begin{innerlist}
    \item
      GNU/Linux user since 1996, experience with OpenBSD and Solaris
    \item
      Experience with Apache, nginx, Cassandra, MySql, PostgreSQL, Bind, Riak,
      Rails, Spring, JPA, Hibernate, RabbitMQ, ZooKeeper, HAProxy etc
\end{innerlist}

\halfblankline

Hardware/Embedded systems:
%
\begin{innerlist}
    \item
      Experience with PCB design and creation in CadSoft Eagle (2-sided),
      digital electronics, low power embedded systems, Microchip PIC 8
      bit family
\end{innerlist}

\halfblankline

\section{Extra- Curricular Activities}
My current passion is rock-climbing, in particular bouldering and 
sport-climbing.

Diving has also been a big interest in my life for the last 7 years and
I enjoy all parts of it, from advanced cave-diving and deep-diving to
educating new divers.

\section{Other}
I speak/write Swedish fluently and have some German skills. I have a
drivers licence.

% \section{References}
% 
% Available upon request.

\vspace{0.1in}
%% \section{More Information}
%% 
%% More information and auxiliary documents can be found at\\%
%% \url{http://www.bolg.se/}.

\end{document}
