%%%%%%%%%%%%%%%%%%%%%%%%%%%%%%%%%%%%%%%%%%%%%%%%%%%%%%%%%%%%%%%%%%%%%%%%
%%%%%%%%%%%%%%%%%%%%%% Simple LaTeX CV Template %%%%%%%%%%%%%%%%%%%%%%%%
%%%%%%%%%%%%%%%%%%%%%%%%%%%%%%%%%%%%%%%%%%%%%%%%%%%%%%%%%%%%%%%%%%%%%%%%

%%%%%%%%%%%%%%%%%%%%%%%%%%%%%%%%%%%%%%%%%%%%%%%%%%%%%%%%%%%%%%%%%%%%%%%%
%% NOTE: If you find that it says                                     %%
%%                                                                    %%
%%                           1 of ??                                  %%
%%                                                                    %%
%% at the bottom of your first page, this means that the AUX file     %%
%% was not available when you ran LaTeX on this source. Simply RERUN  %%
%% LaTeX to get the ``??'' replaced with the number of the last page  %%
%% of the document. The AUX file will be generated on the first run   %%
%% of LaTeX and used on the second run to fill in all of the          %%
%% references.                                                        %%
%%%%%%%%%%%%%%%%%%%%%%%%%%%%%%%%%%%%%%%%%%%%%%%%%%%%%%%%%%%%%%%%%%%%%%%%

%%%%%%%%%%%%%%%%%%%%%%%%%%%% Document Setup %%%%%%%%%%%%%%%%%%%%%%%%%%%%

% Don't like 10pt? Try 11pt or 12pt
\documentclass[10pt]{article}

% The automated optical recognition software used to digitize resume
% information works best with fonts that do not have serifs. This
% command uses a sans serif font throughout. Uncomment both lines (or at
% least the second) to restore a Roman font (i.e., a font with serifs).
\usepackage{times}
\renewcommand{\familydefault}{\sfdefault}

%% This makes font encoding use 8 bits instead of the default 7.
\usepackage[T1]{fontenc}

% The OCR software also has a hard time with italics. These commands get
% rid of the two common ways to italicize text in LaTeX. Get rid of them
% to turn italics back on.
\renewcommand\emph[1]{#1}
\renewcommand\textit[1]{\underline{#1}}

% This is a helpful package that puts math inside length specifications
\usepackage{calc}

% Layout: Puts the section titles on left side of page
\reversemarginpar

%
%         PAPER SIZE, PAGE NUMBER, AND DOCUMENT LAYOUT NOTES:
%
% The next \usepackage line changes the layout for CV style section
% headings as marginal notes. It also sets up the paper size as either
% letter or A4. By default, letter was used. If A4 paper is desired,
% comment out the letterpaper lines and uncomment the a4paper lines.
%
% As you can see, the margin widths and section title widths can be
% easily adjusted.
%
% ALSO: Notice that the includefoot option can be commented OUT in order
% to put the PAGE NUMBER *IN* the bottom margin. This will make the
% effective text area larger.
%
% IF YOU WISH TO REMOVE THE ``of LASTPAGE'' next to each page number,
% see the note about the +LP and -LP lines below. Comment out the +LP
% and uncomment the -LP.
%
% IF YOU WISH TO REMOVE PAGE NUMBERS, be sure that the includefoot line
% is uncommented and ALSO uncomment the \pagestyle{empty} a few lines
% below.
%

%% Use these lines for letter-sized paper
%\usepackage[paper=letterpaper,
%            %includefoot, % Uncomment to put page number above margin
%            marginparwidth=1.2in,     % Length of section titles
%            marginparsep=.05in,       % Space between titles and text
%            margin=1in,               % 1 inch margins
%            includemp]{geometry}

%% Use these lines for A4-sized paper
\usepackage[paper=a4paper,
            %includefoot, % Uncomment to put page number above margin
            marginparwidth=30.5mm,    % Length of section titles
            marginparsep=1.5mm,       % Space between titles and text
            margin=25mm,              % 25mm margins
            includemp]{geometry}

%% More layout: Get rid of indenting throughout entire document
\setlength{\parindent}{0in}

\usepackage[shortlabels]{enumitem}

% Simpler bibsections for CV sections
% (thanks to natbib for inspiration)
%
% * For lists of references with hanging indents and no numbers:
%
%   \begin{bibsection}
%       \item ...
%   \end{bibsection}
%
% * For numbered lists of references (with hanging indents):
%
%   \begin{bibenum}
%       \item ...
%   \end{bibenum}
%
%   Note that bibenum numbers continuously throughout. To reset the
%   counter, use
%
%   \restartlist{bibenum}
%
%   at the place where you want the numbering to reset.

\makeatletter
\newlength{\bibhang}
\setlength{\bibhang}{1em}
\newlength{\bibsep}
 {\@listi \global\bibsep\itemsep \global\advance\bibsep by\parsep}
\newlist{bibsection}{itemize}{3}
\setlist[bibsection]{label=,leftmargin=\bibhang,%
        itemindent=-\bibhang,
        itemsep=\bibsep,parsep=\z@,partopsep=0pt,
        topsep=0pt}
\newlist{bibenum}{enumerate}{3}
\setlist[bibenum]{label=[\arabic*],resume,leftmargin={\bibhang+\widthof{[999]}},%
        itemindent=-\bibhang,
        itemsep=\bibsep,parsep=\z@,partopsep=0pt,
        topsep=0pt}
\let\oldendbibenum\endbibenum
\def\endbibenum{\oldendbibenum\vspace{-.6\baselineskip}}
\let\oldendbibsection\endbibsection
\def\endbibsection{\oldendbibsection\vspace{-.6\baselineskip}}
\makeatother

%%% Setup header and footer (with page number and possible last page)
%
% The first block sets up pages 2--end
% The second block sets up page 1 formatting
%
%%%
%
% NOTE: comment the +LP lines and uncomment the -LP lines to have page
%       numbers without the ``of ##'' last page reference)
%
% NOTE: uncomment the \pagestyle{empty} line to get rid of all page
%       numbers on pages 2--end. To get rid of page numbers on page 1,
%       comment out the \thispagestyle{plain} line on the first page
%       below.
%       (also make sure includefoot is commented out above)
%
\usepackage{fancyhdr,lastpage}
\pagestyle{fancy}
%\pagestyle{empty}      % Uncomment this to get rid of page numbers
\fancyhf{}\renewcommand{\headrulewidth}{0pt}
\fancyfootoffset{\marginparsep+\marginparwidth}
\newlength{\footpageshift}
\setlength{\footpageshift}
          {0.5\textwidth+0.5\marginparsep+0.5\marginparwidth-2in}

%%%% PAGES 2--9 NUMBERING:
%% These two lines put page number in upper-right corner of pages 2--end
\rhead{Jensen-Urstad, p.~\arabic{page} of \protect\pageref*{LastPage}}   % +LP
%\rhead{Pavlic, p.~\arabic{page}}                                 % -LP

%% These lines put page number in bottom (center) of pages 2--end
%\lfoot{\hspace{\footpageshift}%
%       \parbox{4in}{\, \hfill %
%                    \arabic{page} of \protect\pageref*{LastPage} % +LP
%%                    \arabic{page}                               % -LP
%                    \hfill \,}}
%%%% END PAGE 2--9 NUMBERING

%%%% PAGE 1 NUMBERING:
\makeatletter
\let\oldps@plain\ps@plain
\renewcommand{\ps@plain}{\oldps@plain%
\renewcommand{\@evenfoot}{\hspace*{-\footpageshift}\hfil %
    p.~\arabic{page} of \protect\pageref*{LastPage} % +LP
%    p.~\arabic{page}                               % -LP
    \hfil}%
\renewcommand{\@oddfoot}{\@evenfoot}}
\makeatother
%%%% END PAGE 1 NUMBERING

% Finally, give us PDF bookmarks and colored links
%
% NOTE: Some OCR software might be negatively affected by hyperlinks. So
%       most employers recommend the draft option here. Alternatively,
%       making all links black (as opposed to darkblue) should hopefully
%       prevent problems with most OCR.
%
% (to enable hyperlinks and bookmarks, comment out ``draft'' line;
%  to disable hyperlinks and bookmarks, uncomment ``draft'' line)
\usepackage{color,hyperref}
\definecolor{darkblue}{rgb}{0.0,0.0,0.3}
\hypersetup{breaklinks,colorlinks,
            linkcolor=black,urlcolor=black,
            anchorcolor=black,citecolor=black,
            %linkcolor=darkblue,urlcolor=darkblue,
            %anchorcolor=darkblue,citecolor=darkblue,
            %draft
            }

%%%%%%%%%%%%%%%%%%%%%%%% End Document Setup %%%%%%%%%%%%%%%%%%%%%%%%%%%%


%%%%%%%%%%%%%%%%%%%%%%%%%%% Helper Commands %%%%%%%%%%%%%%%%%%%%%%%%%%%%

%%% HEADING AT TOP OF CURRICULUM VITAE

% The title (name) with a horizontal rule under it
% (optional argument typesets an object right-justified across from name
%  as well)
%
% Usage: \makeheading{name}
%        OR
%        \makeheading[right_object]{name}
%
% Place at top of document. It should be the first thing.
% If ``right_object'' is provided in the square-braced optional
% argument, it will be right justified on the same line as ``name'' at
% the top of the CV. For example:
%
%       \makeheading[\emph{Curriculum vitae}]{Your Name}
%
% will put an emphasized ``Curriculum vitae'' at the top of the document
% as a title. Likewise, a picture could be included:
%
%   \makeheading[\includegraphics[height=1.5in]{my_picutre}]{Your Name}
%
% the picture will be flush right across from the name.
\newcommand{\makeheading}[2][]%
        {\hspace*{-\marginparsep minus \marginparwidth}%
         \begin{minipage}[t]{\textwidth+\marginparwidth+\marginparsep}%
             {\large \bfseries #2 \hfill #1}\\[-0.15\baselineskip]%
                 \rule{\columnwidth}{1pt}%
         \end{minipage}}

%%% SECTION HEADINGS

% The section headings. Flush left in small caps down pseudo-margin.
%
% Usage: \section{section name}
\renewcommand{\section}[1]{\pagebreak[3]%
    \vspace{1.3\baselineskip}%
    \phantomsection\addcontentsline{toc}{section}{#1}%
    \noindent\llap{\scshape\smash{\parbox[t]{\marginparwidth}{\hyphenpenalty=10000\raggedright #1}}}%
    \vspace{-\baselineskip}\par}

%%% LISTS

% This macro alters a list by removing some of the space that follows the list
% (is used by lists below)
\newcommand*\fixendlist[1]{%
    \expandafter\let\csname preFixEndListend#1\expandafter\endcsname\csname end#1\endcsname
    \expandafter\def\csname end#1\endcsname{\csname preFixEndListend#1\endcsname\vspace{-0.6\baselineskip}}}

% These macros help ensure that items in outer-type lists do not get
% separated from the next line by a page break
% (they are used by lists below)
\let\originalItem\item
\newcommand*\fixouterlist[1]{%
    \expandafter\let\csname preFixOuterList#1\expandafter\endcsname\csname #1\endcsname
    \expandafter\def\csname #1\endcsname{\csname preFixOuterList#1\endcsname\let\oldItem\item\def\item{\pagebreak[2]\oldItem}}
    \expandafter\let\csname preFixOuterListend#1\expandafter\endcsname\csname end#1\endcsname
    \expandafter\def\csname end#1\endcsname{\let\item\oldItem\csname preFixOuterListend#1\endcsname}}
\newcommand*\fixinnerlist[1]{%
    \expandafter\let\csname preFixInnerList#1\expandafter\endcsname\csname #1\endcsname
    \expandafter\def\csname #1\endcsname{\let\oldItem\item\let\item\originalItem\csname preFixInnerList#1\endcsname}
    \expandafter\let\csname preFixInnerListend#1\expandafter\endcsname\csname end#1\endcsname
    \expandafter\def\csname end#1\endcsname{\csname preFixInnerListend#1\endcsname\let\item\oldItem}}

% An itemize-style list with lots of space between items
%
% Usage:
%   \begin{outerlist}
%       \item ...    % (or \item[] for no bullet)
%   \end{outerlist}
\newlist{outerlist}{itemize}{3}
    \setlist[outerlist]{label=\enskip\textbullet,leftmargin=*}
    \fixendlist{outerlist}
    \fixouterlist{outerlist}

% An environment IDENTICAL to outerlist that has better pre-list spacing
% when used as the first thing in a \section
%
% Usage:
%   \begin{lonelist}
%       \item ...    % (or \item[] for no bullet)
%   \end{lonelist}
\newlist{lonelist}{itemize}{3}
    \setlist[lonelist]{label=\enskip\textbullet,leftmargin=*,partopsep=0pt,topsep=0pt}
    \fixendlist{lonelist}
    \fixouterlist{lonelist}

% An itemize-style list with little space between items
%
% Usage:
%   \begin{innerlist}
%       \item ...    % (or \item[] for no bullet)
%   \end{innerlist}
\newlist{innerlist}{itemize}{3}
    \setlist[innerlist]{label=\enskip\textbullet,leftmargin=*,parsep=0pt,itemsep=0pt,topsep=0pt,partopsep=0pt}
    \fixinnerlist{innerlist}

% An environment IDENTICAL to innerlist that has better pre-list spacing
% when used as the first thing in a \section
%
% Usage:
%   \begin{loneinnerlist}
%       \item ...    % (or \item[] for no bullet)
%   \end{loneinnerlist}
\newlist{loneinnerlist}{itemize}{3}
    \setlist[loneinnerlist]{label=\enskip\textbullet,leftmargin=*,parsep=0pt,itemsep=0pt,topsep=0pt,partopsep=0pt}
    \fixendlist{loneinnerlist}
    \fixinnerlist{loneinnerlist}

%%% EXTRA SPACE

% To add some paragraph space between lines.
% This also tells LaTeX to preferably break a page on one of these gaps
% if there is a needed pagebreak nearby.
\newcommand{\blankline}{\quad\pagebreak[3]}
\newcommand{\halfblankline}{\quad\vspace{-0.5\baselineskip}\pagebreak[3]}

%%% FORMATTING MACROS

% Uses hyperref to link DOI
\newcommand\doilink[1]{\href{http://dx.doi.org/#1}{#1}}
\newcommand\doi[1]{doi:\doilink{#1}}

% For \url{SOME_URL}, links SOME_URL to the url SOME_URL
\providecommand*\url[1]{\href{#1}{#1}}
% Same as above, but pretty-prints SOME_URL in teletype fixed-width font
\renewcommand*\url[1]{\href{#1}{\texttt{#1}}}

% For \email{ADDRESS}, links ADDRESS to the url mailto:ADDRESS
\providecommand*\email[1]{\href{mailto:#1}{#1}}
% Same as above, but pretty-prints ADDRESS in teletype fixed-width font
%\renewcommand*\email[1]{\href{mailto:#1}{\texttt{#1}}}

%\providecommand\BibTeX{{\rm B\kern-.05em{\sc i\kern-.025em b}\kern-.08em
%    T\kern-.1667em\lower.7ex\hbox{E}\kern-.125emX}}
%\providecommand\BibTeX{{\rm B\kern-.05em{\sc i\kern-.025em b}\kern-.08em
%    \TeX}}
\providecommand\BibTeX{{B\kern-.05em{\sc i\kern-.025em b}\kern-.08em
    \TeX}}
\providecommand\Matlab{\textsc{Matlab}}

% Custom hyphenation rules for words that LaTeX has trouble with
\hyphenation{bio-mim-ic-ry bio-in-spi-ra-tion re-us-a-ble pro-vid-er}

%%%%%%%%%%%%%%%%%%%%%%%% End Helper Commands %%%%%%%%%%%%%%%%%%%%%%%%%%%

%%%%%%%%%%%%%%%%%%%%%%%%% Begin CV Document %%%%%%%%%%%%%%%%%%%%%%%%%%%%

\begin{document}
\thispagestyle{plain}
\makeheading[\emph{Curriculum vitae}]{Bj�rn Jensen-Urstad}

\section{Contact Information}

% NOTE: Mind where the & separators and \\ breaks are in the following
%       table. Table is one row made up of three parboxes. The left
%       parbox has address info, the middle parbox has a vertical bar,
%       and the right parbox has phone and electronic contact
%       information.
%
% MACROS: \rcollength is the width of the right column of the table
%             (adjust it to your liking; default is 1.85in).
%         \spacewidth is width of area between left and right boxes.
%         \spacechar is character used to produce perforated vertical
%             boundary between boxes.
%
\newlength{\rcollength}\setlength{\rcollength}{2.5in}%
\newlength{\spacewidth}\setlength{\spacewidth}{20pt}
\newcommand\spacechar{$|$}
%
\begin{tabular}[t]{@{}p{\textwidth-\rcollength-\spacewidth}@{}p{\spacewidth}@{}p{\rcollength}}%

% Address box
\parbox{\textwidth-\rcollength-\spacewidth}{%
Bj�rn Jensen-Urstad\\
Skeppargatan 96\\
115 31 Stockholm\\
Sweden}

% Cheesy perforated vertical bar between boxes
% Shorten by removing \spacechar's
& \parbox{\spacewidth}{\centering \spacechar\\\spacechar\\\spacechar\\\spacechar\\\spacechar} &

% Non-snail-mail contact information
\parbox{\rcollength}{%
\emph{Mobile:} +46-70-0222948\\
\emph{E-mail:} \email{bjorn.jensen.urstad@gmail.com}\\
%%\emph{WWW:} \href{http://www.bernie.com/}{http://www.bolg.se/}\\
\emph{Github:} \href{https://github.com/b3rnie/}{https://github.com/b3rnie/}}

\end{tabular}

%%
%% In modern CV's, it seems like ``Objective'' is frowned upon. Instead,
%% incorporate it into a well-constructed cover letter. The ``More
%% information'' can go at the end of the CV, but it should not distract
%% from the section giving references available to contact.
%%
%
% \section{Objective}
%
% Full-time position that allows for advanced research in electrical and
% computer engineering (communications, control, software, electronics,
% and sustainability), with a particular focus on complex distributed
% systems (i.e., modeling, analysis, design, and verification)
% \begin{innerlist}
%     \item For more information, see \url{http://www.tedpavlic.com/engjobsearch/}
% \end{innerlist}

\section{Qualifications and Interests}

Distributed systems, Erlang/OTP, embedded systems, low power embedded,
software development, scaling, databases, protocols, functional programming,
low level programming, fault tolerance, network protocols, eventual consistency,
software archi\-tecture, message queues

\section{Availability}

\begin{loneinnerlist}
  %%\item Start time is negotiable; may be possible to start immediately
\item Geographic location is flexible, but there is preference
  for Stockholm, Sweden
\end{loneinnerlist}

% \section{Citizenship}
%
% USA

\section{Education}

\href{http://www.kth.com/}{\textbf{Royal Institute of Technology}},
Stockholm, Sweden
\begin{outerlist}

\item[] B.S.,
        \href{http://www.kth.se/}
             {Computer Science}, 2006-2010, specializing in software
             development

\item[] MS.S.,
        \href{http://www.kth.se/}
             {Computer Science}, 2002-2004, four semesters out of 9

\end{outerlist}

\halfblankline

\textbf{Hersby gymnasium}, Liding�, Sweden
\begin{outerlist}
\item[] Upper secondary school, 1997-2000
\end{outerlist}

\section{Professional Experience}


\href{http://www.kivra.se/}{\textbf{Kivra AB}},
Stockholm, Sweden
\begin{outerlist}
\item[] \textit{Developer}%
  \hfill \textbf{March 2013 to September 2013}
  \begin{innerlist}
    \item
      Architecture, development and maintenance of the services that made
      up Kivra's system.
  \end{innerlist}
\end{outerlist}

\halfblankline

\href{http://www.klarna.com/}{\textbf{Klarna AB}},
Stockholm, Sweden
\begin{outerlist}
\item[] \textit{Developer}%
  \hfill \textbf{March 2010 to January 2013}
  \begin{innerlist}
  \item
    Extended an existing transactional database (Mnesia)
    with features to allow parts of the data to be shared (read/written)
    in an eventual consistent manner.
    This involved writing a framework to allow easy transformation
    of keys/values between two very different databses with different
    availability requirements, adding vector clocks to an existing database
    schema and mechanisms to transport/sync data over RabbitMQ.
    The databases are the core of a business with a monthly revenue of
    200 million USD
  \item
    Architecture and implementation of the alpha version of Klarna's checkout
  \item
    Experiments with automatic fraud detection
  \item
    Rewrote and replaced Klarna's old fraud checking system with a DSL
    accessible through a web GUI.
    This gave staff more freedom to express what they want and the
    possibility to change rules without involving developers
  \item
    Lots of minor features for the business such as integrating with new
    credit check agencies, refactoring code
  \end{innerlist}
\end{outerlist}

\halfblankline

\href{http://www.posten.se/}{\textbf{Posten AB}},
Stockholm, Sweden
\begin{outerlist}
\item[] \textit{Mailman}%
  \hfill \textbf{2001 to 2002 and 2004 to 2005}
\end{outerlist}

\halfblankline

\section{Memberships}
\begin{outerlist}
\item[] \textit{Housing Cooperative, Bjornligan}%
  \hfill \textbf{2007-2010}
  \begin{innerlist}
    \item Board of directors and responsible for the economy
  \end{innerlist}
\item[] \textit{EFF (Electronic Frontier Foundation)}%
\hfill \textbf{2010-}
\end{outerlist}
% Add a little space to nudge next ``Submitted Conference Publications'' marginpar
% down to make room for tall ``Submitted Journal Publications''
% marginpar. If there are enough submitted journal publications, this
% space will not be needed (and should be removed).
\vspace{0.1in}

\section{Service}

Written several open-source projects, including:
\begin{innerlist}
  \item Crontab implementation in Erlang
  \item Unique id generator in Erlang
  \item NNTP fetcher/assembler in Perl
  \item Socks 4/5 proxy in C++ (single thread using nonblocking IO)
\end{innerlist}

\halfblankline

\section{Skills}
Software development:
%
\begin{innerlist}
    \item
      Worked with Scrum, Kanban, Git, Subversion, test driven development,
      property based testing
\end{innerlist}

\halfblankline

Computer Programming:
%
\begin{innerlist}
    \item
      Excellent knowledge of Erlang/OTP and functional programming
    \item
      Good knowledge of C, C$+$$+$, Perl, Java, PHP, Ruby
    \item
      Experience with JavaScript, Python, Haskell, Prolog, VHDL, XML/XSLT,
      nios2 and PIC assembly
\end{innerlist}

\halfblankline

Information/Internet Technology:
%
\begin{innerlist}
    \item
      Written my own TCP/IP stack running on a 8 bit microcontroller.
      From transport layer up to application layer (Ethernet, ARP, IP, ICMP,
      TCP, UDP). Implemented a simple webserver and bittorrent tracker
      on top of it
    \item
      Written servers and or clients for FTP, HTTP 1.0/1.1, SMTP, DNS, POP3,
      Jxta, NNTP, Irc, BitTorrent and Socks 4/5
    \item
      Done web development in PHP, Ruby on Rails, Perl, CGI, Yaws
\end{innerlist}

\halfblankline

Operating Systems/Software:
%
\begin{innerlist}
    \item
      GNU/Linux user since 1996, experience with OpenBSD and Solaris
    \item
      Experience with Apache, PostgreSQL, Bind, Riak, Rails, Spring, JPA,
      Hibernate, RabbitMQ, ZooKeeper, HAProxy
\end{innerlist}

\halfblankline

Hardware/Embedded systems:
%
\begin{innerlist}
    \item
      Experience with PCB design and creation in CadSoft Eagle (2-sided),
      digital electronics, low power embedded systems, Microchip PIC 8
      bit family
\end{innerlist}

\halfblankline

\section{Extra- Curricular Activities}
My current passion is rock-climbing, in particular sport-climbing and trad-
climbing.
Diving has also been a big interest in my life for the last 7 years and
I enjoy all parts of it, from advanced cave-diving and deep-diving to
educating new divers.

\section{Other}
I speak/write Swedish fluently and have some German skills. I have a
drivers licence.

\section{References}

Available upon request.

% Add a little space to nudge next ``More Information'' marginpar
% down to make room for tall ``References Available to Contact''
% marginpar.
\vspace{0.1in}
\section{More Information}

More information and auxiliary documents can be found at\\%
\url{http://www.bolg.se/}.

\end{document}

%%%%%%%%%%%%%%%%%%%%%%%%%% End CV Document %%%%%%%%%%%%%%%%%%%%%%%%%%%%%

%----------------------------------------------------------------------%
% The following is copyright and licensing information for
% redistribution of this LaTeX source code; it also includes a liability
% statement. If this source code is not being redistributed to others,
% it may be omitted. It has no effect on the function of the above code.
%----------------------------------------------------------------------%
% Copyright (c) 2007, 2008, 2009, 2010, 2011 by Theodore P. Pavlic
%
% Unless otherwise expressly stated, this work is licensed under the
% Creative Commons Attribution-Noncommercial 3.0 United States License. To
% view a copy of this license, visit
% http://creativecommons.org/licenses/by-nc/3.0/us/ or send a letter to
% Creative Commons, 171 Second Street, Suite 300, San Francisco,
% California, 94105, USA.
%
% THE SOFTWARE IS PROVIDED "AS IS", WITHOUT WARRANTY OF ANY KIND, EXPRESS
% OR IMPLIED, INCLUDING BUT NOT LIMITED TO THE WARRANTIES OF
% MERCHANTABILITY, FITNESS FOR A PARTICULAR PURPOSE AND NONINFRINGEMENT.
% IN NO EVENT SHALL THE AUTHORS OR COPYRIGHT HOLDERS BE LIABLE FOR ANY
% CLAIM, DAMAGES OR OTHER LIABILITY, WHETHER IN AN ACTION OF CONTRACT,
% TORT OR OTHERWISE, ARISING FROM, OUT OF OR IN CONNECTION WITH THE
% SOFTWARE OR THE USE OR OTHER DEALINGS IN THE SOFTWARE.
%----------------------------------------------------------------------%
